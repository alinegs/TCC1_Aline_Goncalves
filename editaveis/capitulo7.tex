\chapter[Conclusão]{Conclusão}

Neste trabalho foi constatado que é possível aplicar uma solução baseada em métodos ágeis e no pensamento \textit{lean} sobre a gestão de um contrato.
 
Para a realização deste trabalho, algumas atividades foram essenciais: a análise de alguns modelos e normas, a revisão da literatura e a definição de um estudo de caso.
 
Com o projeto do estudo de caso, foi possível definir de forma objetiva as questões específicas de pesquisa que procuraram, por sua vez, responder a questão de pesquisa e ao problema definido neste trabalho. Com base nessas questões, os resultados das análises dos dados coletados de questionário, de documentos, do código fonte e de processos foram categorizados em três efeitos: efeitos sobre a entrega de ordens de serviço, efeitos sobre a satisfação do cliente e efeitos sobre a qualidade interna do código fonte.
 
Como unidade de estudo de caso, foi selecionado o contrato do Sistema Integrado de Conhecimento e Gestão (SICG) do Instituto do Patrimônio Histórico e Artístico Nacional (IPHAN). Trata-se de estudo exploratório, onde foram coletados dados qualitativos, que caracterizaram a organização contratante, a empresa contratada, o objeto do contrato e a solução de gestão de contrato definida. Também foram coletados dados quantitativos categorizados nos efeitos mencionados, os quais procuraram responder a questão de pesquisa: \textit{ Como o uso de métodos ágeis e do pensamento \textit{lean} na gestão de contratos de fornecedores de desenvolvimento de \textit{software} influenciaram no resultado final do contrato do ponto de vista do gestor de contrato e do fiscal técnico do contrato, que juntos gerenciam o contrato?}. Com isso, essa influência no resultado final do contrato foi analisada com relação a influência sobre as ordens de serviço, satisfação do cliente e qualidade interna do código fonte.

No que diz respeito aos efeitos sobre a entrega de ordens de serviço, conclui-se que o uso da solução, baseada em métodos ágeis e no pensamento na gestão desse contrato, resultou em uma entrega de uma versão de \textit{software} funcional, no máximo, mensalmente; a entrega de apenas documentação figurou em apenas uma ordem de serviço do projeto, valorizando a fase de execução e entrega de produto funcional; os requisitos atendidos ao longo do projeto foram em torno de 78\%; não houve aplicação de multas no projeto; e o custo final do projeto ultrapassou apenas 1,8\% do custo estimado.
 
No que diz respeito aos efeitos sobre a satisfação do cliente, a área requisitante ficou satisfeita com o produto entregue e a visibilidade do processo por parte do gestor do contrato foi alta, o que possibilitou tomada de decisões rápidas e uma comunicação efetiva com a empresa contratada.
 
No que diz respeito aos efeitos sobre a qualidade interna do código fonte, ao longo das \textit{sprints} do projeto, a qualidade “Excelente” foi maioritária no que diz respeito às 11 métricas analisadas (LOC, ACCM, AMLOC, ACC, ANPM, DIT, LCOM4, NOC, NOM, NPA, RFC) e na percepção da maioria dos envolvidos do projeto a qualidade também foi definida como “Excelente”. No entanto, ao inserirmos a métrica CBO na análise, a qualidade é rotulada como “Ruim”. Uma investigação mais aprofundada poderia esclarecer se tal fator pode ser explicado pelo uso de uma combinação de diversos \textit{frameworks} ocasionando um alto acoplamento do projeto, possibilitando, com isso, a apresentação de diversos cenários de limpeza de código. 

Possivelmente uma solução automatizada de análise estática de código, por exemplo, como a utilizada neste estudo de caso, poderia facilitar a vistoria técnica sobre o código fonte, onde a informação resultante dessa análise pudesse auxiliar a tomada de decisão sobre o faturamento das ordens de serviço. Isto corroboraria com o Art. 25, inciso III, alínea b da Instrução Normativa nº 04 \cite{IN04:2010} que diz para realizar a \textit{"avaliação da qualidade dos serviços realizados ou dos bens entregues e justificativas, de acordo com os Critérios de Aceitação definidos em contrato, a cargo dos Fiscais Técnico e Requisitante do Contrato"}.

Assim, a solução desenvolvida pelo IPHAN, aplicada no projeto SICG, resultou na entrega do \textit{software} requisitado ao final do contratado.  Embora esta solução seja passível de melhorias, pois há indícios de que alguns resultados apresentados não tenham sido totalmente satisfatórios, podemos concluir que o problema de pesquisa deste estudo \textit{"Alguns contratos de desenvolvimento de \textit{software} da organização não resultaram na entrega do \textit{software} requisitado ao final do contrato."}, que ocorria frequentemente no órgão, não ocorreu na gestão de contratos ágil aplicada.

Como analisado no Capítulo 6, a solução de gestão de contratos definida pelo IPHAN não fere o que é determinado na Lei nº 8.666/93 ou na IN 04/2010  ou nos Princípios da APF. Ao mesmo tempo, a solução foi baseada nos valores, princípios e práticas do pensamento \textit{lean} e de metodologias ágeis. As lições aprendidas e a criação de conhecimento fazem parte dos preceitos dos métodos ágeis e do pensamento \textit{lean}, portanto, as recomendações, práticas e princípios que não foram evidenciados no contrato analisado nesse estudo de caso podem ser inseridos nos contratos futuros do órgão.

Como apresentado no Capítulo 5, as principais ameaças a validade de um estudo de caso são: validade do contructo, validade interna, validade externa e confiabilidade. 

Com relação a validade do constructo, para a análise dos efeitos sobre as ordens de serviço, satisfação do cliente e qualidade interna do código fonte, foi utilizada a técnica GQM, a qual direcionou a definição sistemática de problema, objetivos e questões específicas de pesquisa e métricas. Além disso, foram utilizadas múltiplas fontes de evidências e informantes chaves, o que reforçou a validade do constructo. No entanto, as análises da adequação da solução ao Pensamento \textit{Lean}, ao Scrum, a Lei nº 8.666/93, a IN 04/2010 e aos Princípios da APF não foram realizadas a partir de uma técnica formal como o GQM, tais análises foram observadas ao decorrer desse estudo de caso a partir das múltiplas fontes de evidências coletadas. Como trabalhos futuros, pretende-se fortalecer a validade do constructo utilizando a técnica GQM para realização dessas análises de forma mais assertiva.

Com relação a validade interna, ela foi garantida por meio da triangulação de dados e de metodologia para apresentação e análise dos resultados apresentados no Capítulo 6. Os dados foram triangulados por meio de quatro fontes: documentação dos processos do projeto SICG, questionário (Apêndice A), entrevistas informais com os informantes chaves (fiscal técnico e coordenador do projeto) e repositório do projeto (código fonte e documentos). Foram ainda coletados tanto dados qualitativos (percepções dos envolvidos no projeto) quanto dados quantitativos (qualidade do código fonte, documentação e ordens de serviço do projeto), o que figurou a triangulação de metodologia.

Com relação a validade externa,  ela não foi atendida nesse estudo de caso por se tratar de um estudo de caso único com uma única unidade de análise. Além disso, este é um estudo de caso com propósitos exploratórios. Como trabalhos futuros, objetiva-se a ampliação desse estudo para múltiplos casos.

Com relação a confiabilidade, o banco de dados desse estudo de caso foi composto pelas ordens de serviço e atas de reuniões dos Processos nº 01450.011592/2010-30 e nº 01450.000845/2012-10, pelo repositório do projeto, o qual contém o código fonte e a documentação entregue em cada \textit{sprint} do projeto, e pelos dados coletados do questionário localizado no Apêndice A. O banco de dados será disponibilizado juntamente com esse Trabalho de Conclusão de Curso, assim, garantindo a confiabilidade desse estudo de caso. Com isso, qualquer pesquisador que realizar a replicação desse estudo de caso utilizando o mesmo protocolo e as mesmas fontes de dados poderá encontrar os mesmos resultados.
 
Assim, como trabalhos futuros, objetiva-se a validação e refinamentos do protocolo deste estudo de caso, e em seguida, a ampliação deste estudo para outras organizações públicas brasileiras. Além disso, pretende-se realizar um relação mais sistemática entre o uso de métodos ágeis na gestão de contratos de fornecedores de desenvolvimento de \textit{software} e os aspectos da legislação vigente.
 