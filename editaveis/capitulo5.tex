\chapter[Projeto do Estudo de Caso]{Projeto do Estudo de Caso}

Neste capítulo será apresentado o projeto do estudo de caso resultante da fase de Planejamento do Estudo de Caso, o qual contém a definição da unidade de análise,  a identificação do problema, a definição das questões de pesquisa (geral e específicas) e a definição dos objetivos (geral e específicos), assim como considerações sobre as fontes e métodos de coleta de dados e sobre a validade do estudo de caso.

\section[Definição]{Definição}

Neste trabalho foi apresentado um arcabouço teórico relacionado à contratação de serviços de TI em organizações públicas brasileiras. Também foram apresentados os principais conceitos Lean na Manufatura, Lean no Desenvolvimento de \textit{Software}, Metodologias Ágeis.

À luz do levantamento bibliográfico realizado, não foram encontrados estudos que buscassem levantar e analisar os aspectos advindos do uso de metodologias ágeis no contexto da gestão de contratos de fornecedores de desenvolvimento de \textit{software} para organizações públicas brasileiras.


O escopo do estudo de caso está ilustrado na Fig. (17). 
\begin{figure}[H]
		\centering
		\label{fig01}
			\includegraphics[scale=0.6]{figuras/escopoEC.png}
		\caption{Escopo do Estudo de Caso}
\end{figure}

Assim, a proposta deste trabalho consiste na investigação, coleta, análise e discussão dos resultados, de dados de uma contratação de fornecedor de desenvolvimento de \textit{software} pelo Instituto do Patrimônio Histórico e Cultural (IPHAN). A partir da análise  dos dados será realizada uma comparação entre os resultados obtidos com o estudo de caso e os efeitos percebidos pelos principais envolvidos no contrato. O foco será a análise da solução desenvolvida pela organização, a qual é alinhada com os métodos ágeis, com o pensamento lean e com a fase de Gerenciamento do Contrato.


Este trabalho foi estruturado conforme ilustrado na Fig. (18).

\begin{figure}[htb]
		\centering
		\label{fig01}
			\includegraphics[scale=1.0]{figuras/estruturaEstudo.png}
		\caption{Estrutura do Estudo de Caso}
	\end{figure}

O Problema refere-se ao problema de pesquisa identificado. A Questão de Pesquisa refere-se a questão geral de pesquisa que buscará responder o problema. O Objetivo Geral refere-se ao objetivo a ser atingido para que a questão de pesquisa seja respondida. O objetivo geral foi dividido em dois objetivos específicos, relacionados ao processo e ao produto. Para atingir os objetivos específicos e consequentemente responder a questão geral de pesquisa, foram definidas oito questões de pesquisas especificas que serão analisadas no estudo de caso. As questões específicas de pesquisa foram obtidas por meio da técnica GQM.  A definição dessa estrutura está a seguir.

\textbf{Problema:} Alguns contratos de desenvolvimento de software da organização não resultaram na entrega do software requisitado ao final do contrato.

\textbf{Questão de Pesquisa:} Como o uso de métodos ágeis e do pensamento lean na gestão de contratos de fornecedores de desenvolvimento de software influenciaram no resultado final do contrato?

\textbf{Objetivo Geral:} Analisar a influência do uso de métodos ágeis e do pensamento lean no contrato do Sistema Integrado de Conhecimento e Gestão (SICG) com a empresa EGL - Engenharia a partir dos dados coletados da documentação, observação e entrevistas.

\textbf{OE1. Objetivo Específico do Processo:} Analisar a influência do uso de métodos ágeis e do pensamento lean no processo de gestão de contrato do contrato do Sistema Integrado de Conhecimento e Gestão (SICG) com a empresa EGL - Engenharia a partir dos dados coletados da documentação, observação e entrevistas.

\textbf{Questões Específicas do Processo:}


\textbf{QE1.}  Qual a quantidade total de ordens de serviço?

\textbf{Fonte:} Documentação

\textbf{Métrica:} quantidade total de ordens de serviço.
 
\vspace{\onelineskip} 

\textbf{QE2.} Qual a quantidade de ordens de serviço que tiveram entrega de software funcional?

\textbf{Fonte:} Documentação

\textbf{Métrica:} quantidade de ordens de serviço de software.
 
 \vspace{\onelineskip} 

\textbf{QE3.} Qual a quantidade de ordens de serviço que tiverem entrega apenas de documentação?

\textbf{Fonte:} Documentação

\textbf{Métrica:} quantidade de ordens de serviço de documentação.

 \vspace{\onelineskip} 
 
\textbf{QE4.} Qual a proporção de entrega de software funcional?

\textbf{Fonte:} Documentação

\textbf{Métrica:} quantidade de ordens de serviço de software que tiveram entrega de software funcional /quantidade total de ordens de serviço.

 \vspace{\onelineskip} 

\textbf{QE5.} Qual a duração média de entrega de software funcional?

\textbf{Fonte:}Documentação

\textbf{Métrica:} duração total das sprints/duração total das sprints que tiveram de entrega de software funcional.

 \vspace{\onelineskip} 

\textbf{QE6.} Qual a quantidade de ordens de serviço que não teve entrega de software funcional e de documentação?

\textbf{Fonte:} Documentação

\textbf{Métrica:} quantidade de ordens de serviço sem software e documentação.

 \vspace{\onelineskip} 
 
\textbf{QE7.} Qual a porcentagem de requisitos atendidos em cada ordem de serviço?

\textbf{Fonte:} Documentação

\textbf{Métrica:} (requisitos atendidos/requisitos pedidos) * 100.
 
 \vspace{\onelineskip} 

\textbf{QE8.} Quantas multas foram aplicadas no contrato?

\textbf{Fonte:} Documentação

\textbf{Métrica}: quantidade de multas.

 \vspace{\onelineskip}  

\textbf{QE9.} O quanto de visibilidade do que estava sendo feito o gestor do negócio teve durante o contrato?

\textbf{Fonte:} Gestor do negócio, gestor do contrato, coordenador do projeto.

\textbf{Métrica:} alto, médio ou baixo. 
 
 \vspace{\onelineskip} 

\textbf{QE10.} Qual o nível de satisfação com o software entregue ao final do contrato?

\textbf{Fonte:} Gestor do negócio, gestor do contrato, coordenador do projeto.

\textbf{Métrica:} muito satisfeito, satisfeito, neutro, insatisfeito, muito insatisfeito.
 
 \vspace{\onelineskip} 

\textbf{OE2. Objetivo Específico do Produto:}Analisar a qualidade do código fonte com o uso de métodos ágeis e do pensamento lean na gestão do contrato do contrato do Sistema Integrado 
de Conhecimento e Gestão (SICG) com a empresa EGL - Engenharia a partir dos dados coletados da documentação, observação e entrevistas.

\textbf{Questão Específicas do Produto:}

\textbf{QE8.} Qual a qualidade interna do produto entregue até o momento ?

\textbf{Fonte:} Código

\textbf{Métrica:} bom, excelente, regular e preocupante.


\section[Fonte e Método Coleta de Dados]{Fonte e Método de Coleta de Dados}

Os dados foram coletados por meio de entrevistas informais, observações, questionários e por meio da análise de documentos processuais da organização e da base de código fonte de um contrato disponibilizado pelo órgão: o contrato do Sistema Integrado de Conhecimento e Gestão (SICG) com a empresa EGL - Engenharia, no qual foram utilizadas metodologias ágeis para gestão do contrato.Os questionários tinham o objetivo de coletar dados qualitativos e
quantitativos a respeito da organização contratante, do gestor de negócio (cliente) e da empresa contratada no que diz respeito a estrutura organizacional, experiência prévia, satisfação, opniões, percepções e etc. O dados de observação e entrevistas
complementaram os questionários sob o ponto de vista qualitativo.Os dados quantitativos
sobre a execução do processo (solução) e a qualidade do código fonte foram coletados de 20 Sprints do projeto foram coletados da documentação e do código fonte do contrato SICG .

\textbf{Documentos}
\begin{itemize}
\item Processo Pai.
\item Processo Filho.
\item Código Fonte.
\item Backlog do Produto.
\end{itemize}

\textbf{Questionários}
\begin{itemize}
\item Questionário para o IPHAN: tem como objetivo coletar informações dos envolvidos no projeto por parte do IPHAN, tais como o gestor do contrato e o coordenador do projeto.
\item Questionário para o Gestor do Negócio: tem como objetivo coletar dados informações do gestor do negócio, o qual representa o papel de cliente do projeto. 
\item Questionário para a Empresa: tem como objetivo coletar informações dos envolvidos no projeto por parte da empresa contratada.
\end{itemize}

\section[Validade]{Validade}

As principais ameaças aos estudos de caso aplicáveis a este estudo de caso
são mencionadas por Yin (2009). Dentre elas, destaca-se a confiabilidade dos dados
coletados e dos resultados obtidos. Para Seaman (1999), o uso de várias fontes de dados
e métodos de coleta permite a triangulação, uma técnica para confirmar se os resultados
de diversas fontes e de diversos métodos convergem. Dessa forma é possível aumentar a
validade interna do estudo e aumentar a força das conclusões. Nesta pesquisa houve
triangulação de dados  e de metodologia. A triangulação de dados se deu
pelo uso de base de documentos e código organizacionais, questionários e entrevistas para coletar dados. A triangulação
de métodos ocorreu pelo uso de métodos de coleta quantitativos e qualitativos.