\chapter[O Estudo de Caso]{O Estudo de Caso}


\section[A Organização]{A Organização}

O órgão escolhido, IPHAN, possui uma força de trabalho atuante na área de TI de apenas 8 funcionários, dos quais apenas 3 trabalham diretamente com sistemas. O perfil dessa equipe é apresentado na Tab. (3). Devido ao número reduzido de servidores disponíveis na área de TI do órgão, frequentemente uma mesma pessoa acaba desempenhando diferentes papeis requeridos pela Instrução Normativa MP/SLTI Nº04/2010.

\begin{table}[H]
\center
\footnotesize
\begin{tabular}{|c|c|c|}
\hline
\textbf{Área}          & \textbf{Perfil}   & \textbf{Quantidade} \\ \hline
TI Geral               & Coordenador de Tecnologia da Informação   & 1                   \\ \hline
Infraestrutura         & Analista de Tecnologia da Informação   & 2                   \\ \hline
Sistemas               & Analista de Tecnologia da Informação   & 2                   \\ \hline
\multirow{3}{*}{Apoio} & Analista de Tecnologia da Informação    & 1                   \\ \cline{2-3} 
\multicolumn{1}{|l|}{} & Servidor do Ministério da Ciência, Tecnologia e Inovação & 1                   \\ \cline{2-3} 
\multicolumn{1}{|l|}{} & Servidor do IPHAN & 1                   \\ \hline
\end{tabular}
\caption{Perfil da Equipe}
\end{table}

O contexto atual do órgão foi identificado por meio da aplicação da técnica de entrevista semi-estruturada. A estrutura da entrevista pode ser encontrada no Apêndice I -  Roteiro de Entrevista.

Os fatores mais significantes que são gerenciados pela área de TI do órgão, segundo o entrevistado, são:
\begin{itemize}
\item Atender as demandas para desenvolvimento de sistemas (sistema novo, manutenção, documentação).
\item Controlar de qualidade de sistemas.
\item Possuir medições de sistemas.
\end{itemize}

Por meio da entrevista foram identificados alguns problemas, dentre eles:
\begin{itemize}
\item Alguns contratos foram encerrados sem haver entrega de \textit{software};
\item Havia faturamento de Ordens de Serviço apenas com entrega de documentação;
\item Incapacidade de aferir a qualidade interna do produto;
\item As mudanças de requisitos geravam impacto no tempo de execução do projeto, ocasionando constantes atrasos.
\end{itemize}

A partir dos problemas identificados, ainda conforme o entrevistado, as principais motivações para o uso de metodologias ágeis na gestão de contratos foram: aumentar o volume de entregas \textit{software}; prover maior visibilidade do processo e do produto para o do gestor de negócio; empoderar o gestor do contrato sobre a gerência dos requisitos, de forma a eliminar  a interferência direta do gestor de contrato e, por fim, perceber de forma mais rápida se o projeto terá sucesso ou não, pois os riscos inerentes à contratação poderiam ser identificados com antecedência.

De acordo com a Instrução Normativa MP/SLTI Nº04/2010, a fase de Gerenciamento de Contrato deve conter as seguintes etapas: início do contrato; encaminhamento formal de ordem de serviço ou fornecimento de bens  monitoramento da execução; e transição contratual e/ou encerramento do contrato. Todas estas etapas podem estão contempladas no MIDAS, fazendo com que ele seja aderente ao normativo e adequado para o estudo de caso deste trabalho.

De acordo com o entrevistado, as metas norteadoras para a elaboração do MIDAS foram:
\begin{itemize}
\item Ser aderente à legislação pertinente;
\item Entregar \textit{software} mais rapidamente;
\item Focar na gestão do contrato e na definição de uma metodologia de gestão de demandas;
\item Não focar em dizer como a empresa deveria desenvolver o \textit{software}, ou seja, não definir metodologia de desenvolvimento de \textit{software};
\item Satisfazer as necessidades do cliente.
\end{itemize}

Outros instrumentos contratuais que foram modificados com o uso do MIDAS foram a forma de pagamento e a aplicação de multas. Em relação a esta, diferentemente das outras formas de gestão de contrato utilizadas anteriormente, onde as multas eram progressivamente aplicadas, por exemplo, sobre a não entrega de documentação somente além de deter caráter meramente punitivo. Com o MIDAS, passou-se a considerar o maturidade e crescimento da empresa no contrato e as multas eram somente aplicadas se não houvesse entrega de \textit{software}. O faturamento das ordens de serviço era executado a cada entrega, final de \textit{sprint}, com duração mensal, o que mantinha o fluxo de caixa da empresa contratada sempre ativo.

\section[Caracterização da Solução]{Caracterização da Solução}

Para desenvolver um modelo de contratação de fornecedores de \textit{software} baseado em Scrum e Kanban, o IPHAN definiu alguns procedimentos que deveriam ser feitos com o foco na minimização dos riscos da execução contratual e na obtenção do sucesso no contrato de terceirização. O \textit{framework} utilizado não é considerado o mais relevante, mas sim os valores e princípios do Manifesto Ágil, além do atendimento à legislação vigente. 

As metodologias ágeis foram utilizadas como o meio para atingir o sucesso ou para identificar de forma rápida os riscos iminentes. O sucesso contratual pode ser entendido como aquele contrato que atende às necessidades do órgão, com sistemas, sem comprometer o erário (tesouro público). Assim, para atingir sucesso em um contrato é preciso que pelos menos esses três procedimentos sejam realizados: \cite{parente}:
\begin{itemize}
\item Definir premissas nos artefatos desde o planejamento da contratação;
\item Alinhar diretrizes e condições com a Direção de TI;
\item Convalidar com a Alta Administração, ou seja, validar e sustentar essas diretrizes durante o contrato.
\end{itemize}

Com isso, foram definidas algumas premissas que devem orientar o planejamento e execução do contrato. A saber:  \cite{parente}:
\begin{itemize}
\item O órgão não deve definir, ou exigir, o uso de Metodologia Ágil da entidade contratada. Não defina Metodologia de Desenvolvimento de Software (MDS), mas sim a
forma de gerenciar as demandas (ordens de serviço), os produtos que devem ser entregues e seus critérios de aceitação. 
\item A recontagem de Pontos por Função nos moldes do roteiro do SISP com metodologia ágil que pode mudar constantemente é um risco. É preciso alterar o percentual definido para a alteração, manutenção ou refatoração de uma funcionalidade, definir corretamente o conceito de manutenção evolutiva, refatoração e alteração de requisito e evidenciar no processo o custo de uma alteração e fazer com que o gestor negocial, que pediu a alteração, assine a ordem de serviço e ateste a nota fiscal;
\item Só abra uma Ordem de Serviço (OS) por vez e por projeto. Pode-se ter várias OS abertas com a mesma Contratada, porém, será uma OS para cada projeto e uma OS por \textit{Sprint}; Além disso, nunca comece oficialmente a próxima demanda sem receber ou finalizar a demanda anterior. Caso uma OS não estiver atendendo o que foi solicitado, ou por uma mudança negocial essa OS não for necessária, cancele-a e abra outra ordem de serviço que atenda a nova exigência do gestor contratual;
\item Não gerencie atrasos ou defeitos. No fim da \textit{Sprint}, receba o que estiver pronto, mesmo que não seja tudo que foi solicitado. Se nada foi entregue é uma ausência de entrega, não existe atraso, a \textit{Sprint} é considerada perdida. O produto não entregue ou com defeito dever voltar para fila de demandas e entrará na próxima OS ou \textit{Sprint} se o gestor negocial a quiser novamente, nunca aceite que corrijam um produto com defeito dentro da mesma OS;
\item Entenda a demanda antes de executá-la. É preciso planejar, pelo menos, com quantas ordens de serviço o projeto será validado, qual o processo de negócio que será desenvolvido, como será feita a gestão de demandas e qual será a demanda da próxima \textit{Sprint} ou OS;
\item Não aceite documentos sem sistemas. É importante ter em mente que não deve-se aceitar entregas apenas de documentação sem um produto funcional;
\item Acredite na evolução da empresa. No começo, a empresa contratada poderá não conseguir entregar o que foi solicitado ou entregar um produto funcional, no entanto, progressivamente ela irá se adequar ao processo e evoluir. 
\end{itemize} 

Com essas premissas definidas o órgão construiu um Kanban para auxiliar a Gestão de Demandas. 

O Kanban definido pelo IPHAN possui quatro colunas ou raias e está ilustrado na Fig. (12).

\begin{figure}[H]
		\centering
		\label{fig05}
			\includegraphics[scale=0.5]{figuras/kanbanIPHAN1.png}
		\caption{Quadro Kanban  \cite{parente}}
\end{figure}

A primeira raia do Kanban diz respeito aos itens que estão no estado “Preparado”. A condição de transição para esta raia pode ser feita da forma que o órgão quiser (Fig. 13). Por exemplo, os itens com mais prioridade podem ser os primeiros a irem para esta coluna. É importante que a definição de “Preparado” e a definição de “Pronto” estejam bem claras para todos os envolvidos.

\begin{figure}[H]
		\centering
		\label{fig06}
			\includegraphics[scale=0.5]{figuras/kanbanIPHAN2.png}
		\caption{Transição para a raia Preparado \cite{parente}}
\end{figure}

A transição de um item da raia “Preparado” para a raia “OS Aberta” ocorre na abertura de uma ordem de serviço (Fig. 14). Ao observar o processo do MIDAS, percebe-se que essa transição ocorre após o planejamento da \textit{sprint}, no subprocesso \textit{Sprint}, onde uma ordem de serviço de desenvolvimento é aberta  com os itens que devem ser desenvolvidos para aquela \textit{sprint} e o desenvolvimento é iniciado. 

\begin{figure}[H]
		\centering
		\label{fig07}
			\includegraphics[scale=0.5]{figuras/kanbanIPHAN3.png}
		\caption{Transição entre raias \cite{parente}}
\end{figure}

A transição da raia “OS Aberta” para a raia “Homologando” ocorre quando o Termo de Recebimento Provisório é emitido (Fig. 14). Ao se observar o subprocesso de Realizar Ateste Técnico, percebe-se que essa transição ocorre na atividade “Receber Produtos”. Esta tem como entrada a ordem de serviço da fase e como saída o termo de recebimento provisório. Com a emissão do termo de recebimento provisório, os produtos recebidos entram na processo de homologação. 

A transição da raia “Homologando” para a raia “Pronto” ocorre quando o Termo de Recebimento Definitivo é emitido, ou seja, quando todos os produtos que foram anteriormente entregues são verificados e aprovados (Fig. 14). Para tanto são aferidos a aderência aos padrões técnicos e aos requisitos a partir de uma análise técnica e negocial, realizadas conjuntamente pelo Fiscal do Contrato e o Gestor de Negócio. Se forem detectados defeitos nos produtos entregues ou se eles forem rejeitados ou tiverem necessidade de refatoração, eles retornam para a fila de demandas, iniciando novamente o ciclo. A sinalização de rejeitado ou \textit{bug} diz respeito a funcionalidade que foi rejeitada por não atender o que foi pedido tanto funcionalmente quanto tecnicamente. A sinalização de refatoração diz respeito a mudança que é pedida em uma funcionalidade depois de ela já ter sido implementada. Para que uma funcionalidade entre nessa sinalização é preciso que o gestor de negócio assuma a responsabilidade pelos impactos que a mudança causará no custo, tempo e escopo.

Vale ressaltar que é importante que o trabalho em progresso (WIP) seja limitado conforme o que é conceituado no método Kanban. O IPHAN definiu um limite de 200 Pontos por Função por ciclo de trabalho (Fig. 15). 

\begin{figure}[h]
		\centering
		\label{fig08}
			\includegraphics[scale=0.5]{figuras/kanbanIPHAN4.png}
		\caption{Limitação de WIP \cite{parente}}
\end{figure}

É importante sempre valorizar a entrega de produto funcional e não pagar por apenas documentação. Assim, o IPHAN dividia a forma de pagamento da contratada em percentuais, de acordo com a fase, valorizando a fase de execução, como ilustrado na Tab. (2).


\begin{table}[H]
\center
\footnotesize
\begin{tabular}{|p{6cm}|p{6cm}|}
  \hline
   \textbf{Fase} & \textbf{Percentual de Pagamento}\\
    \hline
   Planejamento (1 vez) & 5\%\\
   \hline    
   Execução (n vezes) & 80\%\\
    \hline
   Encerramento (1 vez) & 15\%\\
   \hline
\end{tabular}
\caption{Formas de Pagamento}
\end{table}


Outra técnica importante que foi construída diz respeito a parelização das atividades (Fig. 16). Enquanto uma ordem de serviço está na etapa de homologação, outra ordem de serviço pode ser preparada, evitando que o fluxo do processo pare e haja desperdício. 

\begin{figure}[H]
		\centering
		\label{fig09}
			\includegraphics[scale=0.5]{figuras/kanbanIPHAN5.png}
		\caption{Parelização de Atividades \cite{parente}}
\end{figure}

O Kanban evidencia a aderência de utilização de métodos ágeis. A paralelização das atividades, que evita desperdícios de trabalho, evidencia a aderência ao Lean no Desenvolvimento de \textit{Software}. Assim, conclui-se que as premissas utilizadas como base para o desenvolvimento do Kanban foram baseadas tanto nos princípios ágeis quanto nos princípios do Lean. Após alguns meses de aplicação dessa solução o órgão deu início a construção da Metodologia IPHAN de Gestão de Demandas de Desenvolvimento Ágil de \textit{Software} (MIDAS). Uma parte do MIDAS pode ser visualizada no Anexo B. 


\section[Caracterização do Contrato]{Caracterização do Contrato}


\section[Análise dos Dados]{Análise dos Dados}

\subsection[Efeitos sobre a entrega de ordens de serviço]{Efeitos sobre a entrega de ordens de serviço}

\subsection[Efeitos sobre a satisfação do cliente]{Efeitos sobre a satisfação do cliente}

\subsection[Efeitos sobre a qualidade do código]{Efeitos sobre a qualidade do código}

A qualidade do código fonte do SICG pode ser avaliada por meio de métricas. Neste trabalho foram selecionadas as métricas de código fonte levantadas e categorizadas por MEIRELLES: métricas de tamanho e complexidade e métricas de orientação à objetos. 

As métricas de tamanho e complexidade selecionadas são: 

\textbf{LOC (Lines of Code):} número de Linhas de Código foi uma das primeiras métricas
utilizadas para medir o tamanho de um software. São contadas apenas as linhas
executáveis, ou seja, são excluídas linhas em branco e comentários (JONES,
1991).

 \vspace{\onelineskip} 

\textbf{ACCM (Average Cyclomatic Complexity per Method):} média da Complexidade
Ciclomática por Método mede a complexidade dos métodos ou funções
de um programa. Essa métrica pode ser representada através de um grafo de fluxo
de controle (MCCABE, 1976). O uso de estruturas de controle, tais como, if, else,
while aumentam a complexidade ciclomática de um método.

 \vspace{\onelineskip} 

As métricas de orientação à objetos selecionadas são:

\textbf{ACC (Afferent Connections per Class):} conexões Aferentes por Classe é o número
total de classes externas de um pacote que dependem de classes de dentro desse
pacote. Quando calculada no nível da classe, essa medida também é conhecida como
Fan-in da classe, medindo o número de classes das quais a classe é derivada e, assim,
valores elevados indicam uso excessivo de herança múltipla (MCCABE; DREYER;
WATSON, 1994) (CHIDAMBER; KEMERER, 1994).

 \vspace{\onelineskip} 

\textbf{RFC (Response For a Class):} respostas para uma Classe é número de métodos
dentre todos os métodos que podem ser invocados em resposta a uma mensagem
enviada por um objeto de uma classe (SHARBLE; COHEN, 1993).

 \vspace{\onelineskip} 

\textbf{LCOM4 (Lack of Cohesion in Methods):} falta de Coesão entre Métodos. Originalmente
proposto por Chidamber e Kemerer (1994) como LCOM não teve uma
grande aceitabilidade. Após críticas e sugestões a métrica foi revisada por Hitz e
Montazeri (1995), que propôs a LCOM4. Para calcular LCOM4 de um módulo, é
necessário construir um gráfico não-orientado em que os nós são os métodos e atributos
de uma classe. Para cada método, deve haver uma aresta entre ele e um outro
método ou variável que ele usa. O valor da LCOM4 é o número de componentes
fracamente conectados nesse gráfico.

 \vspace{\onelineskip} 

\textbf{NOM (Number of Methods):} número de Métodos é usado para medir o tamanho
das classes em termos das suas operações implementadas. Essa métrica é usada para
ajudar a identificar o potencial de reúso de uma classe. Em geral, as classes com
um grande número de métodos são mais difíceis de serem reutilizadas, pois elas são
propensas a serem menos coesas (LORENZ; KIDD, 1994).

 \vspace{\onelineskip} 

\textbf{DIT (Depth of Inheritance Tree):} profundidade da Árvore de Herança é o número
de superclasses ou classes ancestrais da classe sendo analisada. São contabilizadas
apenas as superclasses do sistema, ou seja, as classes de bibliotecas não são
contabilizadas. Nos casos onde herança múltipla é permitida, considera-se o maior
caminho da classe até uma das raízes da hierarquia. Quanto maior for o valor DIT,
maior é o número de atributos e métodos herdados, e, portanto,maior é a complexidade
(SHIH et al., 1997).

 \vspace{\onelineskip} 

\textbf{NOC (Number of Children):} número de Filhos é o número de subclasses ou classes
filhas que herdam da classe analisada (ROSENBERG; HYATT, 1997). Deve se ter
cautela ao modificar classes com muitos filhos, pois uma simples modificação de
assinatura de um método, pode criar uma mudança em muitas classes.

 \vspace{\onelineskip} 

De forma a agregar os valores comuns, Meirelles (2013) classificou as métricas de código-fonte nos
seguintes intervalos: muito frequente, frequente, pouco frequente e não frequente.
Para simplificar o entendimento das métricas de código-fonte, renomeou-se os
intervalos tal como a Tabela 4. 

\begin{table}[!ht]
	\begin{center}
	 \begin{tabular}{|l|l|}
		\hline
		Intervalo de Frequência & Intervalos Qualitativos \\ \hline
		Muito Frequente & Excelente \\ \hline
		Frequente       & Bom       \\ \hline
		Pouco Frequente & Regular   \\ \hline
		Não Frequente   & Ruim      \\ \hline
		\end{tabular}
		\caption{Nome dos Intervalos de Frequência}
		\label{nomes}
		\end{center}
		\end{table}


Posteriormente, apresenta-se a Tabela 3, decorrente do
estudo de Meirelles (2013), com os intervalos encontrados para C++ e Java. Com isso, os intervalos apresentados na Tabela 5 foram os utilizados como os indicadores
de qualidade de código fonte na análise deste estudo de caso.

\begin{table}[!ht]
	\begin{center}
	\begin{tabular}{ |l|l|l|l| }
		\hline
		Métrica & Intervalo Qualitativo & Java & C++ \\ \hline
		%-------------------------------
		\multirow{4}{*}{LOC} 
		 & Excelente & [de 0 a 33] & [de 0 a 31] \\
		 & Bom & [de 34 a 87] & [de 32 a 84] \\
		 & Regular & [de 88 a 200] & [de 85 a 207] \\
		 & Ruim & [acima de 200] & [acima de 207] \\ \hline
		 %---------------------------------

		 %-------------------------------
		\multirow{4}{*}{ACCM} 
		 & Excelente & [de 0 a 2,8] & [de 0 a 2,0] \\
		 & Bom & [de 2,9 a 4,4] & [de 2,1 a 4,0] \\
		 & Regular & [de 4,5 a 6,0] & [de 4,1 a 6,0] \\
		 & Ruim & [acima de 6] & [acima de 6] \\ \hline
		 %---------------------------------


		%-------------------------------
		\multirow{4}{*}{ACC} 
		 & Excelente & [de 0 a 1] & [de 0 a 2,0] \\
		 & Bom & [de 1,1 a 5] & [de 2,1 a 7,0] \\
		 & Regular & [de 5,1 a 12] & [de 7,1 a 15] \\
		 & Ruim & [acima de 12] & [acima de 15] \\ \hline
		 %---------------------------------


		%-------------------------------
		\multirow{4}{*}{RFC} 
		 & Excelente & [de 0 a 9] & [de 0 a 29] \\
		 & Bom & [de 10 a 26] & [de 30 a 64] \\
		 & Regular & [de 27 a 59] & [de 65 a 102] \\
		 & Ruim & [acima de 59] & [acima de 102] \\ \hline
		 %---------------------------------

		 %-------------------------------
		\multirow{4}{*}{LCOM4} 
		 & Excelente & [de 0 a 3] & [de 0 a 5] \\
		 & Bom & [de 4 a 7] & [de 6 a 10] \\
		 & Regular & [de 8 a 12] & [de 11 a 14] \\
		 & Ruim & [acima de 12] & [acima de 14] \\ \hline
		 %---------------------------------

		 %-------------------------------
		\multirow{4}{*}{NOM} 
		 & Excelente & [de 0 a 8] & [de 0 a 10] \\
		 & Bom & [de 9 a 17] & [de 11 a 17] \\
		 & Regular & [de 18 a 27] & [de 18 a 26] \\
		 & Ruim & [acima de 27] & [acima de 26] \\ \hline
		 %---------------------------------

		 %-------------------------------
		\multirow{4}{*}{DIT} 
		 & Excelente & [de 0 a 2] & [de 0 a 1] \\
		 & Bom & [de 3 a 4] & [de 2 a 3] \\
		 & Regular & [de 5 a 6] & [de 3 a 4] \\
		 &
		  Ruim & [acima de 6] & [acima de 4] \\ \hline
		 %---------------------------------

		%-------------------------------
		\multirow{4}{*}{NOC} 
		 & Excelente & [de 0 a 1] & [0] \\
		 & Bom & [de 1 a 2] & [1] \\
		 & Regular & [de 2 a 3] & [de 1 a 2] \\
		 & Ruim & [acima de 3] & [acima de 2] \\ \hline
		 %---------------------------------

	\end{tabular}
	\caption{Intervalos das Métricas para Java e C++}
	\label{metrics}
	\end{center}
	\end{table}

De forma geral, o processo realizado para a análise do código fonte do SICG foi retirado do trabalho desenvolvido por BAUFAKER, que  consiste em: executar a ferramenta Analizo sobre o código fonte do projeto SICG, o qual resultará em um arquivo .csv com os resultados númericos das métricas selecionadas, transformar o arquivo .csv em json e inserir esse arquivo na arquitetura desenvolvida por BAUFAKER na ferramenta Pentaho. Ao final desse processo, temos como resultado os gráficos para análise. 


\section[Resultados]{Resultados}

