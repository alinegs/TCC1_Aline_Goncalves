\begin{resumo}[Abstract]
 \begin{otherlanguage*}{english}
 
In Brazil, the situation of public contracting software development suppliers has changed over time. Some public organizations have agreed on the understanding that the contractual instruments in force today, do not prioritize the delivery of software, nor its internal quality and business value. This contributes to finish projects without success. Currently, some organizations of the Federal Public Administration begin investments  to adopt contracting software development services using agile methods. However, each of these organizations have experienced different experiences and have shared common difficulties with regard to the limitations imposed by the normative of software contracting. The purpose of this study was to conduct a case study to analyze the influence of the use of agile methods and lean thinking in the management of public demands of software development. The research was descriptive, with technical support from the case study and the strategy used was an exploratory case study on the management solution for agile contracts defined by a Brazilian government organization in order to analyze the influence of the same on the result of contract with regard to the effects on the delivery of service orders, on customer satisfaction and the quality of the internal source. As future work, the objective is the validation and refinement of the protocol of this case study, and then the expansion of this study to other Brazilian public organizations.

   \vspace{\onelineskip}
 
   \noindent 
   \textbf{Key-words}:  Contracting. Software. Organizations. Lean. Agile methods. Management. Service Orders. Customer Satisfaction. Internal quality.
 \end{otherlanguage*}
\end{resumo}
