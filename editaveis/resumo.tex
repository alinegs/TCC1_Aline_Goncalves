\begin{resumo}
No Brasil, o cenário das contratações públicas de fornecedores de desenvolvimento de \textit{software} vem passando por mudanças. Algumas organizações públicas tem compactuado entendimento de que os instrumentos contratuais, hoje em vigor, não priorizam a entrega de \textit{software}, tampouco sua qualidade interna e valor de negócio. Isso contribui para que projetos terminem sem sucesso, o que onera os cofres públicos. Atualmente, algumas organizações da Administração Pública Federal iniciam investimentos para adotar contratações de serviços de desenvolvimento de \textit{software} utilizando métodos ágeis. Contudo, cada uma dessas organizações têm vivenciado diferentes experiências e vêm compartilhando dificuldades comuns no que diz respeito às limitações impostas pelo normativo de contratação de \textit{software}. Neste trabalho será apresentado um referencial teórico que envolve: contratações públicas brasileiras de \textit{software}; o pensamento Lean e métodos ágeis. O objetivo será realizar uma investigação empírica do uso de tais abordagens na gestão de contratos públicos de terceirização de software. Para tanto, será realizado um estudo de caso em um órgão público específico a fim de analisar os efeitos do uso de tais métodos, principalmente, sobre a entrega de ordens de serviço, sobre a satisfação do cliente e sobre a qualidade interna do código fonte.

\vspace{\onelineskip}
    
 \noindent
 \textbf{Palavras-chaves}: contratações; software; lean; kanban; scrum; métodos ágeis; gerenciamento; gestão; produção; software.
\end{resumo}
