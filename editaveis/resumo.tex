\begin{resumo}
 O cenário de Contratações de Fornecedores de Desenvolvimento de Software
está sofrendo uma mudança no que diz respeito a metodologia que é utilizada na gestão de contratos.
As instituições públicas brasileiras estão buscando novas formas de gerir seus contratos, pois
perceberam que o produto principal de um software, o código, não estava sendo entregue
e muitos projetos nesta área terminam sem sucesso. Neste trabalho será apresentado um referencial 
teórico que envolve principalmente Contratações, o Pensamento Lean, o Kanban e o Scrum, e tem como objetivo a realização de um estudo de caso em um órgão público específico e espera-se ter como resultado a resposta às questões de pesquisas
que serão definidas para análise, principalmete, do uso de Kanban na gestão de contratos.

 \vspace{\onelineskip}
    
 \noindent
 \textbf{Palavras-chaves}: contratações. software. lean. kanban. scrum.
\end{resumo}
