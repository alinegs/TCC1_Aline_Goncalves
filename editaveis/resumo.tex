\begin{resumo}
No Brasil, o cenário das contratações públicas de fornecedores de desenvolvimento de \textit{software} vem passando por mudanças. Algumas organizações públicas tem compactuado entendimento de que os instrumentos contratuais, hoje em vigor, não priorizam a entrega de \textit{software}, tampouco sua qualidade interna e valor de negócio. Isso contribui para que projetos terminem sem sucesso, o que onera os cofres públicos. Atualmente, algumas organizações da Administração Pública Federal iniciam investimentos para adotar contratações de serviços de desenvolvimento de \textit{software} utilizando métodos ágeis. Contudo, cada uma dessas organizações têm vivenciado diferentes experiências e vêm compartilhando dificuldades comuns no que diz respeito às limitações impostas pelo normativo de contratação de \textit{software}.  O objetivo deste trabalho foi realizar uma investigação empírica do uso do pensamento \textit{lean} e de métodos ágeis na gestão de contratos públicos de terceirização de software. A pesquisa foi descritiva, com apoio da técnica de estudo de caso. Foi realizado um estudo de caso exploratório sobre a solução de gestão de contratos ágeis definida por um órgão público brasileiro a fim de analisar a influência da mesma sobre o resultado final do contrato no que diz respeito aos efeitos sobre a entrega de ordens de serviço, sobre a satisfação do cliente e sobre a qualidade interna do código fonte. Como trabalhos futuros, objetiva-se a validação e refinamentos do protocolo deste estudo de caso, e em seguida, a ampliação deste estudo para outras organizações públicas brasileiras.

\vspace{\onelineskip}
    
 \noindent
 \textbf{Palavras-chaves}:  Contratações. \textit{Software}. Organizações. \textit{Lean}. Métodos Ágeis. Gestão. Ordens de Serviço. Satisfação do Cliente. Qualidade Interna.
\end{resumo}
