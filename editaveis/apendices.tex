\begin{apendicesenv}

\partapendices

\chapter{Roteiro da Entrevista}

Data: 

Hora de Início:

Hora de Término:

Procedimentos:

\begin{enumerate}
\item Apresentação do objetivo da reunião
\item Caracterização do órgão público
\item Identificação dos Participantes
\item Questões
\begin{enumerate} 
\item Qual a quantidade de funcionários disponíveis na área de TI do órgão?
\item Quais os problemas que vocês identificaram com o uso de metodologias tradicionais na gestão de contratos e que motivou o uso de metodologias ágeis?
\item Quais foram as metas definidas para a construção do MIDAS?
\item Quais foram os principais beneficios advindos do uso de metodologias ágeis, em especial do uso do Kanban?
\item Como era a aferição da qualidade com o uso de metodologias tradicionais e com o uso de metodologias ágeis antes da definição do ateste técnico no MIDAS?
\item Como era a aplicação de multas com o uso de metodologias tradicionais? E como ficou com o uso de metodologias ágeis?
\item Dentre os critérios que podemos utilizar para analisar a diferença entre o uso de uma metodologia em contraposição ao uso da outra metodologia, quais que você considera mais relevantes e possíveis de serem analisados com os documentos que serão fornecidos?
\begin{enumerate}
\item Tempo de entrega de software funcional.
\item	O custo do software.
\item	Número de ordens de serviços. 
\item	Satisfação do cliente.
\item	Tempo para realização de uma mudança de software.
\item Quantidade de documentação.
\item Quantidade de documentação desnecessária.
\item Qualidade do produto.
\end{enumerate}
\end{enumerate}
\end{enumerate}

Tem mais sugestões de critérios?


\chapter{Segundo Apêndice}

Texto do segundo apêndice.

\end{apendicesenv}
